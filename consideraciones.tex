\section{Consideraciones}

\begin{itemize}
	\item Se asume lógica de cortocircuito para todos los algoritmos.

	\item Algunas complejidades y sus justificaciones son compartidas (ya que la función es prácticamente idéntica, solo llama a una o dos auxiliares o solo realiza alguna otra acción en $\Theta(1)$), así que en algunos casos se omiten y se asume que se usa la próxima complejidad y justificación disponible

	\item La aridad de \texttt{EntrenadoresPosibles} se modifica ya que tomaba un conjunto de jugadores por parámetro. Se asume que el funcionamiento es igual a llamar a esa función con todos los jugadores válidos.

	\item Para ser consistente con el cambio a iteradores en la funcion \texttt{Jugadores}, también se devuelve un iterador en \texttt{Expulsados}.
\end{itemize}