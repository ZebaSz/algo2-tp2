\Section{Modulo ColaPrioridad}

\begin{Interfaz}

	\textbf{parámetros formales}\hangindent=2\parindent\\
	\parbox{1.7cm}{\textbf{géneros}} \TipoVariable{$\alpha$}\\
	\parbox[t]{1.7cm}{\textbf{función}}\parbox[t]{\textwidth-2\parindent-1.7cm}{
		\InterfazFuncion{Copiar}{\In{s}{$\alpha$}}{$\alpha$}
		{$res \igobs s$}
		[$\Theta(copy(s))$]
		[función de copia de $\alpha$'s.]
	}

	\textbf{géneros}: \TipoVariable{\colaPrior, \itColaPrior}.

	\TituloDis{Operaciones básicas de ColaPrioridad(\texorpdfstring{$\alpha$}{α})}

	\InterfazFuncion{Vacía}{}{colaPrior}
	{$res$ $\igobs$ vacía}
	[$O(1)$]
	[creación de una cola de prioridad vacía.]

	\InterfazFuncion{Encolar}{\Inout{c}{\colaPrior}, \In{e}{$\alpha$}}{nat}
	[$c \igobs c_0$]
	{$res \igobs$ encolar($e, c_0$)}
	[$O(copy(e) + )$]
	[encola una copia de e]

	\InterfazFuncion{Vacía?}{\In{c}{\colaPrior}}{bool}
	{$res \igobs$ vacía?($c$)}
	[$O(1)$]
	[devuelve true si, y solo si, la cola no contiene elementos.]

	\InterfazFuncion{Próximo}{\In{c}{\colaPrior}}{$\alpha$}
	[$\not vacía(c)$]
	{$res \igobs$ próximo?($c$)}
	[$O(copy(res))$]
	[devuelve el elemento mas chico]

	\InterfazFuncion{Desencolar}{\Inout{c}{\colaPrior}{}
	[$c \igobs c_0$]
	{$res \igobs$ desencolar($c_0$)}
	[$O()$]
	[elimina el elemento mas chico de la cola]


\end{Interfaz}