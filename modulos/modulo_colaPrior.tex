\section{Modulo ColaPrioridad(\texorpdfstring{$\alpha$}{α})}

Esta cola de prioridad genérica se representa sobre un conjunto, el cual a su vez se implementa sobre un arbol de búsqueda autobalanceado (AVL). Gracias a esto, no solo podemos realizar cualquier función en tiempo logarítmico o menor, sino que también nos permite agregar la función Borrar, que nos permite eliminar cualquier elemento de la cola. Para establecer correctamente la prioridad, se requiere que $\alpha$ tenga definida una relación de orden estricta.

Se usa $\#C$ para denotar la cantidad de elementos de la cola, $elem_1$ y $elem_2$ para representar elementos cualesquiera de la cola (aquellos que más cueste comparar) y $newMin$ para representar el nuevo mínimo luego de haber removido el mínimo actual.

\begin{Interfaz}

	\textbf{parámetros formales}\hangindent=2\parindent\\
	\parbox{1.7cm}{\textbf{géneros}}$\alpha$\\
	\parbox[t]{1.7cm}{\textbf{función}}\parbox[t]{.5\textwidth-\parindent-1.7cm}{
		\InterfazFuncion{$\bullet < \bullet$}{\In{a_1}{$\alpha$}, \In{a_2}{$\alpha$}}{bool}
		{$res \igobs (a_1 < a_2)$}
		[$\Theta(isLess(a_1, a_2))$]
		[comparación de $\alpha$'s]
	}
	\parbox[t]{1.7cm}{\textbf{función}}\parbox[t]{.5\textwidth-\parindent-1.7cm}{
		\InterfazFuncion{Copiar}{\In{a}{$\alpha$}}{$\alpha$}
		{$res \igobs a$}
		[$\Theta(copy(a))$]
		[función de copia de $\alpha$'s]
	}

	\textbf{se explica con}: \tadNombre{Cola de prioridad($\alpha$)}, \tadNombre{Iterador Unidireccional($\alpha$)}.

	\textbf{géneros}: \TipoVariable{colaPrior($\alpha$), itColaPrior($\alpha$)}.

	\TituloDis{Operaciones básicas de ColaPrioridad($\alpha$)}

	\InterfazFuncion{Vacía}{}{colaPrior}
	{$res$ $\igobs$ vacía}
	[$O(1)$]
	[crea una cola de prioridad vacía.]

	\InterfazFuncion{Encolar}{\Inout{c}{colaPrior}, \In{a}{$\alpha$}}{}
	[$c \igobs c_0$]
	{$res \igobs$ encolar($a, c_0$)}
	[$O(log(\#C) \times isLess(a, elem_1) + copy(a))$]
	[encola un elemento.]
	[se almacena una copia de $a$.]

	\InterfazFuncion{Vacía?}{\In{c}{colaPrior}}{bool}
	{$res \igobs$ vacía?($c$)}
	[$O(1)$]
	[devuelve \texttt{true} si y solo si la cola no contiene elementos.]

	\InterfazFuncion{Próximo}{\In{c}{colaPrior}}{$\alpha$}
	[$\neg$ vacía(c)]
	{res $\igobs$ próximo?($c$)}
	[$O(1)$]
	[devuelve el elemento mas chico de la cola.]
	[$res$ no es modificable.]

	\InterfazFuncion{Desencolar}{\Inout{c}{colaPrior}}{}
	[$\neg$ vacía(c) $\land$ $c \igobs c_0$]
	{$res \igobs$ desencolar($c_0$)}
	[$O(log(\#C) \times isLess(elem_1, elem_2))$]
	[desencola el elemento mas chico de la cola.]

	\InterfazFuncion{Borrar}{\Inout{c}{colaPrior}, \In{a}{$\alpha$}}{colaPrior}
	[$c \igobs c_0$]
	{$res \igobs$ borrar($c_0$, a)}
	[$O(log(\#C) \times isLess(elem_1, elem_2))$]
	[desencola un elemento particular de la cola si el mismo pertenecía.]

  \TituloDis{Especificación de las operaciones auxiliares utilizadas en la interfaz}

  \begin{tad}{Cola de prioridad($\alpha$) Extendido}
    \parskip=0pt
    \tadExtiende{\tadNombre{Cola de prioridad($\alpha$)}}
    
    \tadTitulo{otras operaciones (exportadas)}
    \tadOperacion{borrar}{colaPrior($\alpha$), $\alpha$}{nat}{}
    \tadAxiomas
    \tadAxioma{borrar(c,a)}{\IF vacia?(c) $\oluego$ THEN 
    	c
    ELSE 
    	{\IF proximo(c) = a THEN
    		desencolar(c)
    	ELSE
    		encolar(proximo(c), borrar(desencolar(c),a))
    	FI}
    FI}
  \end{tad}

	\TituloDis{Operaciones básicas del iterador}

	El iterador que presentamos no permite modificar la cola recorrida.

	\InterfazFuncion{CrearIt}{\In{c}{colaPrior}}{itColaPrior}
	{true}
	[$O(1)$]
	[crea un iterador unidireccional de los elementos.]
	[el iterador se invalida si y solo si se elimina el siguiente elemento del iterador. Además, siguientes($res$) podría cambiar completamente ante cualquier operación que modifique el conjunto.]

	\InterfazFuncion{HayMas}{\In{it}{itColaPrior}}{bool}
	{$res$ $\igobs$ hayMas?($it$)}
	[$O(1)$]
	[devuelve \texttt{true} si y sólo si en el iterador todavía quedan elementos para avanzar.]

	\InterfazFuncion{Actual}{\In{it}{itColaPrior}}{$\alpha$}
	[HayMas?($it$)]
	{$res$ $\igobs$ Actual($it$)}
	[$O(1))$]
	[devuelve el elemento siguiente a la posición del iterador.]

	\InterfazFuncion{Avanzar}{\Inout{it}{itColaPrior}}{}
	[$it = it_0$ $\land$ HayMas?($it$)]
	{$it$ $\igobs$ Avanzar($it_0$)}
	[$O(log(\#c))$]
	[avanza a la posición siguiente del iterador.]

\end{Interfaz}

\begin{Representacion}
	\TituloDis{Representación de colaPrior}

	\begin{Estructura}{colaPrior($\alpha$)}[eCola]
		\begin{Tupla}[eCola]
			\tupItem{conjElem}{conjOrd($\alpha$)}
			\tupItem{menor}{$\alpha$}
		\end{Tupla}
	\end{Estructura}

	\TituloDis{Invariante de Representación de ColaPrior($\alpha$)}
	
	\begin{enumerate}

		\item El menor es efectivamente el menor del conjunto de elementos

	\end{enumerate}

	\Rep[eCola][e]{vacio?(e.conjElem) $\oluego$ e.menor $\igobs$ minimo(e.conjElem)}

	~

	\TituloDis{Función de Abstracción}

	\Abs[eCola]{colaPrior}[e]{d}{conjACola(e.conjElem)}
	
	~

	\tadOperacion{conjACola}{conj($\alpha$)}{colaPrior($\alpha$)}{}

	\tadAxioma{conjACola(conj)}{\IF $\emptyset$?(conj) THEN vacia ELSE encolar(dameUno(conj),conjACola(sinUno(conj))) FI}

	~

	\TituloDis{Representación del iterador}

	\begin{Estructura}{itColaPrior($\alpha$)}[itConjOrd($\alpha$)]
	\end{Estructura}

	\TituloDis{Invariante de Representación del iterador}

	\Rep[eItCola][it]{true}

	\TituloDis{Función de Abstracción del iterador}

	\Abs[eItCola]{itUni($\alpha$)}[e]{d}
	{TODO}

\end{Representacion}

\begin{Algoritmos}
	\TituloDis{Algoritmos de ColaPrioridad($\alpha$)}

	\begin{algorithm}[H]
		\NoCaptionOfAlgo
		\caption{\algoritmo{iVacia}{}{eCola}}
		res.conjElem $\leftarrow$ Vacio()\OdeLinea{1}
	\end{algorithm}

	\complejidad: $O(1)$

	\justifcomp: solo se inicializa el conjunto vacío.

	~

	\begin{algorithm}[H]
		\NoCaptionOfAlgo
		\caption{\algoritmo{iEncolar}{\Inout{c}{eCola}, \In{a}{$\alpha$}}{}}
		$\alpha$ : min $\leftarrow$ c.menor\OdeLinea{1}
		c.conjElem $\leftarrow$ Agregar(c.conjElem, a)\OdeLinea{log($\#C$) \times isLess(a, elem_1) + copy(a)}
		\If{a $<$ min}{
			c.menor $\leftarrow$ a\OdeLinea{copy(a)}
		}
	\end{algorithm}

	\complejidad: $O(log(\#C) \times isLess(a, elem_1) + copy(a))$

	\justifcomp: se utilizan operaciones del conjunto. También se copia el elemento aparte en caso de ser un nuevo mínimo.

	~

	\begin{algorithm}[H]
		\NoCaptionOfAlgo
		\caption{\algoritmo{iVacia?}{\In{c}{eCola}}{bool}}
		res $\leftarrow$ Vacio?(c.conjElem)\OdeLinea{1}
	\end{algorithm}

	\complejidad: $O(1)$

	\justifcomp: por la manera en la que se representa, si el conjunto de elementos esta vacío, la cola también

	~

	\begin{algorithm}[H]
		\NoCaptionOfAlgo
		\caption{\algoritmo{iProximo}{\In{c}{eCola}}{$\alpha$}}
		res $\leftarrow$ c.menor\OdeLinea{1}
	\end{algorithm}

	\complejidad: $O(1)$

	\justifcomp: conseguimos el minimo del conjunto en O(1) gracias a la copia que tenemos guardada.

	~

	\begin{algorithm}[H]
		\NoCaptionOfAlgo
		\caption{\algoritmo{iDesencolar}{\Inout{c}{eCola}}{}}
		Borrar(c, c.menor)\OdeLinea{log(\#C) \times isLess(elem_1, elem_2) + copy(newMin)}
	\end{algorithm}

	\complejidad: $O(log(\#C) \times isLess(elem_1, elem_2) + copy(newMin))$

	\justifcomp: se utilizan operaciones del conjunto. También se busca y copia el nuevo mínimo.

	~

	\begin{algorithm}[H]
		\NoCaptionOfAlgo
		\caption{\algoritmo{iBorrar}{\Inout{c}{eCola}, \In{a}{$\alpha$}}{}}
		c.conjElem $\leftarrow$ Borrar(e, c.conjElem)\OdeLinea{log(\#C) \times isLess(elem_1, elem_2)}
		\If{a = menor \textbf{and not} Vacio?(c.conjElem)}{
			c.menor $\leftarrow$ Minimo(c.conjElem)\OdeLinea{log(\#C) \times isLess(elem_1, elem_2) + copy(newMin)}
		}
	\end{algorithm}

	\complejidad: $O(log(\#C) \times isLess(elem_1, elem_2) + copy(newMin))$

	\justifcomp: se utilizan operaciones del conjunto. También se busca y copia el nuevo mínimo de ser necesario.

	~

	\TituloDis{Algoritmos del iterador}

	\begin{algorithm}[H]
		\NoCaptionOfAlgo
		\caption{\algoritmo{CrearIt}{\In{c}{eCola}}{eItCola}}
		res $\leftarrow$ CrearIt(c.conjElem)\OdeLinea{1}
	\end{algorithm}

	\complejidad: $O(1)$

	\justifcomp: como el iterador consiste en exponer el iterador del conjunto interno, comparte las mismas complejidades. Los demás algorítmos funcionan igual a los de dicho iterador.

\end{Algoritmos}