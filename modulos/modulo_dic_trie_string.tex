\newcommand\dicString{diccString($\alpha$)}

\section{Módulo Diccionario (String, \texorpdfstring{$\alpha$}{α}) sobre Trie}

Descsripcion TODO

Usaremos $copy(s)$ para denotar el costo de copiar el elemento $s \in \alpha$, y llamaremos $|c_{max}|$ a la longitud de la clave más larga.

\subsection{Interfaz}

\textbf{parámetros formales}\hangindent=2\parindent\\
\parbox{1.7cm}{\textbf{géneros}} \TipoVariable{$\alpha$}\\
%% TODO agregué esto por las dudas, supongo que es necesario, revisar
\parbox[t]{1.7cm}{\textbf{función}}\parbox[t]{\textwidth-2\parindent-1.7cm}{
	\InterfazFuncion{Copiar}{\In{s}{$\alpha$}}{$\alpha$}
	{$res \igobs s$}
	[$\Theta(copy(s))$]
	[función de copia de $\alpha$'s]
	[]
}

\textbf{se explica con}: \tadNombre{Diccionario(String,$\alpha$)}.

\textbf{géneros}: \TipoVariable{\dicString, itDiccString($\alpha$)}.

~

\subsubsection{Operaciones básicas de Diccionario (String, \texorpdfstring{$\alpha$}{α}) sobre Trie}

\InterfazFuncion{CrearDiccionario}{}{\dicString}
[true]
{$res$ $\igobs$ vacío()}
[$O(1)$]
[Creación de un diccionario vacío]
[]

~

\InterfazFuncion{Definir}{\Inout{d}{\dicString}, \In{c}{string}, \In{s}{$\alpha$}}{}
[$d \igobs d_0$]
{$d \igobs$ definir($c, s, d_0$)}
[$O(|c_{max}| + copy(s))$]
[Define sobre la clave $c$ el signigicado $s$ en el diccionario $d$]
[Se almacena una copia de $s$]

~

\InterfazFuncion{Definido?}{\In{d}{\dicString}, \In{c}{string}}{bool}
[true]
{$res \igobs$ def?($c, d$)]}
[$O(|c_{max}|)$]
[Devuelve true si la clave $c$ esta definida en el diccionario.]
[]

~

\InterfazFuncion{Obtener}{\In{d}{\dicString}, \In{c}{string}}{$\alpha$}
[def?($c, d$)]
{alias($res \igobs$ obtener($c$, $d$))}
[$O(|c_{max}|)$]
[devuelve el significado de la clave $c$ en $d$]
[$res$ es modificable si y sólo si $d$ es modificable]

~

\InterfazFuncion{Borrar}{\Inout{d}{\dicString}, \In{c}{string}}{}
[$d \igobs d_0 \land$ def?($c, d$)]
{$d \igobs$ borrar($c, d_0$)}
[$O(|c_{max}|)$]
[Borra la clave $c$ y su significado del diccionario $d$]
[]

~

%%TODO ITERADOR DE CLAVES

\subsubsection{Representación de Diccionario (String, \texorpdfstring{$\alpha$}{α}) sobre Trie)}

\begin{Estructura}{ Diccionario (String, $\alpha$) sobre Trie }[estr]
	\begin{Tupla}[estr]
		\tupItem{raiz}{puntero(Nodo)}
	\end{Tupla}

	~

	\begin{Tupla}[Nodo]
		\tupItem{hijos}{arreglo[256] de puntero(Nodo)}
		\tupItem{valor}{$\alpha$}
		\tupItem{contieneValor}{bool}
	\end{Tupla}

\end{Estructura}

\subsubsection{Invariante de Representación}

\Rep[estr][e]{
	hijoValido(e.raiz)
}

\tadOperacion{hijoValido}{puntero(Nodo)}{bool}{}

\tadAxioma{hijoValido(nodo)}{
	nodo == NULL $\oluego$ ($\forall i \in [0..256)$) (hijoValido(nodo.hijos[i]))
}

\subsubsection{Función de Abstracción}

%%TODO se indefine si raiz es NULL arreglarlo

\Abs[estr]{\dicString}[e]{d}
{($\forall s$: string)
(def?(s, d) = definido?(e.raiz) $\yluego$
(def?(s, d) $\implies$ obtener(s, d) = obtener(e.raiz, s, 0)))
}


\tadOperacion{definido?}{puntero(Nodo)/n, string/c, nat/i}{bool}{n $\neq$ NULL $\land$ i $\leq$ longitud(c)}
\tadOperacion{obtener}{puntero(Nodo)/n, string/c, nat/i}{$\alpha$}{n $\neq$ NULL $\land$ i $\leq$ longitud(c) $\yluego$ definido?(n, c, i)}

\tadAxioma{definido?(nodo, clave, i)}
{\IF i $<$ Longitud(nodo.hijos) THEN
	nodo.hijos[ord(clave[i])] != NULL $\yluego$ definido?(nodo.hijos[ord(clave[i])], clave, i+1)
ELSE
	nodo.contieneValor
FI}

\tadAxioma{obtener(nodo, clave, i)}
{\IF i $<$ Longitud(nodo.hijos) THEN
	obtener(nodo.hijos[ord(clave[i])], clave, i+1)
ELSE
	nodo.valor
FI}

\subsection{Algoritmos}

\begin{algorithm}[H]
	\SetAlgoLined
	\NoCaptionOfAlgo
	\caption{\algoritmo{iCrearDiccionario}{}{estr}}
	\For{$i \leftarrow 0$ \KwTo $255$}{
		res.raiz[i] $\leftarrow$ NULL\;
	}
\end{algorithm}




