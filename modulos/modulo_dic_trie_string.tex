\newcommand\dicString{diccString($\alpha$)}

\section{Módulo Diccionario (String, $\alpha$) sobre Trie}

Descsripcion TODO

\subsection{Interfaz}

\textbf{parametros formales}

\textbf{géneros}: \TipoVariable{$\alpha$}.

\textbf{se explica con}: \tadNombre{Diccionario(String,$\alpha$)}.

\textbf{géneros}: \TipoVariable{\dicString, itDiccString($\alpha$)}.

~

\subsubsection{Operaciones básicas de Diccionario (String, $\alpha$) sobre Trie}

\InterfazFuncion{Vacío}{}{\dicString}
[true]
{$res$ $\igobs$ vacío()}
[]%%Complejidad O(1)
[Creación de un diccionario vacío]
[]

~

\InterfazFuncion{Definir}{\Inout{d}{\dicString}, \In{c}{string}, \In{s}{$\alpha$}}{}
[$d \igobs d_0$]
{$d \igobs$ definir($c, s, d_0$)}
[]%%O(de la clave mas larga)
[Define sobre la clave $c$ el signigicado $s$ en el diccionario $d$]
[]%%c y s por copia???????

~

\InterfazFuncion{Definido?}{\In{d}{\dicString}, \In{c}{string}}{bool}
[true]
{$res \igobs$ def?($c, d$)]}
[]%%O(de la clave mas larga)
[Devuelve true si la clave $c$ esta definida en el diccionario.]
[]

~

\InterfazFuncion{Obtener}{\In{d}{\dicString}, \In{c}{string}}{$\alpha$}
[def?($c, d$)]
{alias($res \igobs$ obtener($c$, $d$))}
[]%%O(de la clave mas larga)
[devuelve el significado de la clave $c$ en $d$]
[$res$ es modificable si y sólo si $d$ es modificable]

~

\InterfazFuncion{Borrar}{\Inout{d}{\dicString}, \In{c}{string}}{}
[$d \igobs d_0 \land$ def?($c, d$)]
{$d \igobs$ borrar($c, d_0$)}
[]%%O(de la clave mas larga)
[Borra la clave $c$ y su significado del diccionario $d$]
[]

~
%%TODO CLAVES

\subsubsection{Representación de Diccionario (String, $\alpha$) sobre Trie)}

\begin{Estructura}{ Diccionario (String, $\alpha$) sobre Trie }[estr]
	\begin{Tupla}[estr]
		\tupItem{raiz}{vector(puntero(Nodo))}
	\end{Tupla}

	~

	\begin{Tupla}[Nodo]
		\tupItem{hijos}{vector(puntero(Nodo))}
		\tupItem{valor}{$\alpha$}
		\tupItem{contieneValor}{bool}
	\end{Tupla}

\end{Estructura}

\subsubsection{Invariante de Representación}

\Rep[estr][e]{
	longitud(e.raiz) == 27 $\land$
	hijosValidos(e.raiz)
}

\tadOperacion{esValido}{puntero(Nodo)/nodo}{bool}{}
\tadOperacion{hijosValidos}{vector(puntero(Nodo))/hijos}{bool}{}

\tadAxioma{esValido(nodo)}{ %%aca iria un if preguntando si el nodo es null??????
	longitud(nodo.hijos) == 27 $\land$
	hijosValidos(nodo.hijos) 
}
%%TODO
\tadAxioma{hijosValidos(hijos)}{ %%Preguntar por punteros
	vacio?(hijos) $\oluego$ (esValido(dameUno(hijos)) $\land$ hijosValidos(sinUno(hijos))
}

\subsubsection{Función de Abstracción}

\Abs[estr]{\dicString}[e]{d}{TODO} %%TODO

\subsection{Algoritmos}

\begin{algorithm}[H]
	\SetAlgoLined
	\NoCaptionOfAlgo
	\caption{\algoritmo{Vacío}{}{estr}}
	res.raiz $\leftarrow$ ArreglosDeNulls(27)\;
\end{algorithm}




