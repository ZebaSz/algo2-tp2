\newcommand\dicString{diccString($\alpha$)}
\newcommand\itDicString{itDiccString($\alpha$)}
\renewcommand{\labelenumi}{(\Roman{enumi})}

\section{Módulo DiccionarioString(\texorpdfstring{$\alpha$}{α})}

El diccionario se representa con un trie, que permite lectura, inserción y modificación en $\Theta(|clave|)$, donde \texttt{clave} es la clave consultar o modificar.

Las claves se guardan al mismo tiempo en un Conjunto Lineal, siempre con inserción rápida ya que al insertar en el trie podemos saber si la clave existía de antemano.

Al tener que mantener las copias de las claves, remover un elemento cuesta $O(|c_{max}| * \#c)$ (ya que debe removerse del conjunto de claves).

Usaremos $copy(s)$ para denotar el costo de copiar el elemento $s \in \alpha$. Llamaremos $|c_{max}|$ a la longitud de la clave más larga, y $\#c$ a la cantidad de claves definidas. Se asume que la complejidad de comparar dos Strings es $O(|c_{max}|)$.

\begin{Interfaz}

	\textbf{parámetros formales}\hangindent=2\parindent\\
	\parbox{1.7cm}{\textbf{géneros}} \TipoVariable{$\alpha$}\\
	\parbox[t]{1.7cm}{\textbf{función}}\parbox[t]{\textwidth-2\parindent-1.7cm}{
		\InterfazFuncion{Copiar}{\In{s}{$\alpha$}}{$\alpha$}
		{$res \igobs s$}
		[$\Theta(copy(s))$]
		[función de copia de $\alpha$'s.]
	}

	\textbf{se explica con}: \tadNombre{Diccionario($\kappa$,$\sigma$)}, \tadNombre{Iterador Bidireccional($\alpha$)}.

	\textbf{géneros}: \TipoVariable{\dicString, \itDicString}.

	\textbf{servicios usados}: \tadNombre{Conjunto Lineal($\alpha$)}


	\TituloDis{Operaciones básicas de DiccionarioString(\texorpdfstring{$\alpha$}{α})}

	\InterfazFuncion{CrearDiccionario}{}{\dicString}
	{$res$ $\igobs$ vacío()}
	[$O(1)$]
	[creación de un diccionario vacío.]

	\InterfazFuncion{Definir}{\Inout{d}{\dicString}, \In{c}{string}, \In{s}{$\alpha$}}{}
	[$d \igobs d_0$]
	{$d \igobs$ definir($c, s, d_0$)}
	[$O(|c_{max}| + copy(s))$]
	[define sobre la clave $c$ el signigicado $s$ en el diccionario $d$.]
	[se almacena una copia de $c$ y de $s$.]

	\InterfazFuncion{Definido?}{\In{d}{\dicString}, \In{c}{string}}{bool}
	{$res \igobs$ def?($c, d$)]}
	[$O(|c_{max}|)$]
	[devuelve true si la clave $c$ esta definida en el diccionario.]

	\InterfazFuncion{Obtener}{\In{d}{\dicString}, \In{c}{string}}{$\alpha$}
	[def?($c, d$)]
	{alias($res \igobs$ obtener($c$, $d$))}
	[$O(|c_{max}|)$]
	[devuelve el significado de la clave $c$ en $d$]
	[$res$ es modificable si y sólo si $d$ es modificable.]

	\InterfazFuncion{Borrar}{\Inout{d}{\dicString}, \In{c}{string}}{}
	[$d \igobs d_0 \land$ def?($c, d_0$)]
	{$d \igobs$ borrar($c, d_0$)}
	[$O(|c_{max}| * \#c)$]
	[borra la clave $c$ y su significado del diccionario $d$.]


	\TituloDis{Operaciones del iterador}

	El iterador que presentamos no permite modificar el diccionario recorrido.

	\InterfazFuncion{CrearIt}{\In{d}{DiccString($\alpha$)}}{\itDicString}
	{alias(esPermutacion?(SecuSuby($res$), $d.claves$)) $\land$ vacia?(Anteriores($res$))}
	[$O(1)$]
	[crea un iterador bidireccional del diccionario usando el iterador del conjunto de sus claves.]
	[el iterador se invalida si y sólo si se elimina el elemento siguiente del iterador. Además, anteriores($res$) y siguientes($res$) podrían cambiar completamente ante cualquier operación que modifique $d$ sin utilizar las funciones del iterador.]

	\InterfazFuncion{HaySiguiente}{\In{it}{\itDicString}}{bool}
	{$res$ $\igobs$ haySiguiente?($it$)}
	[$O(1)$]
	[devuelve \texttt{true} si y sólo si en el iterador todavía quedan elementos para avanzar.]

	\InterfazFuncion{HayAnterior}{\In{it}{\itDicString}}{bool}
	{$res$ $\igobs$ hayAnterior?($it$)}
	[$O(1)$]
	[devuelve \texttt{true} si y sólo si en el iterador todavía quedan elementos para retroceder.]

	\InterfazFuncion{Siguiente}{\In{it}{\itDicString}}{tupla(String, $\alpha$)}
	[HaySiguiente?($it$)]
	{alias($res$ $\igobs$ Siguiente($it$))}
	[$O(|c_{max}|))$]
	[devuelve el elemento siguiente a la posición del iterador, como tupla clave-valor.]
	[$res$ no es modificable.]

	\InterfazFuncion{Anterior}{\In{it}{\itDicString}}{tupla(String, $\alpha$)}
	[HayAnterior?($it$)]
	{alias($res$ $\igobs$ Anterior($it$))}
	[$O(|c_{max}|)$]
	[devuelve el elemento anterior a la posición del iterador, como tupla clave-valor.]
	[$res$ no es modificable.]

	\InterfazFuncion{Avanzar}{\Inout{it}{\itDicString}}{}
	[$it = it_0$ $\land$ HaySiguiente?($it$)]
	{$it$ $\igobs$ Avanzar($it_0$)}
	[$O(1)$]
	[avanza a la posición siguiente del iterador.]

	\InterfazFuncion{Retroceder}{\Inout{it}{\itDicString}}{}
	[$it = it_0$ $\land$ HayAnterior?($it$)]
	{$it$ $\igobs$ Retroceder($it_0$)}
	[$O(1)$]
	[retrocede a la posición anterior del iterador.]

\end{Interfaz}

\begin{Representacion}

	\TituloDis{Representación de DiccionarioString(\texorpdfstring{$\alpha$}{α})}

	\begin{Estructura}{DiccionarioString($\alpha$)}[dicStr]

		\begin{Tupla}[dicStr]
			\tupItem{raiz}{puntero(Nodo)}
			\tupItem{claves}{conj(String)}
		\end{Tupla}

		~

		\begin{Tupla}[Nodo]
			\tupItem{hijos}{arreglo[256] de puntero(Nodo)}
			\tupItem{valor}{$\alpha$}
			\tupItem{contieneValor}{bool}
		\end{Tupla}

	\end{Estructura}

	\TituloDis{Invariante de Representación del Diccionario}

	\begin{enumerate}

		\item La raíz del trie es un nodo válido no nulo que no guarda valor
		% TODO tal vez si queremos que se haga nulo o que guarde valor

		\item Los punteros son únicos (se referencian desde un único punto del trie)
		% TODO agregar a Rep

	\end{enumerate}


	\Rep[dicStr][e]{
		(e.raiz $\neq$ NULL) $\yluego \neg$(e.raiz.contieneValor)
	}

	\TituloDis{Función de Abstracción del Diccionario}

	\Abs[dicStr]{\dicString}[e]{d}
	{($\forall s$: string)
	(def?(c, d) = definido?(e.raiz, c, 0) $\yluego$ \\
	(def?(c, d) $\impluego$ obtener(c, d) = obtener(e.raiz, c, 0)))
	}

	~

	\tadOperacion{definido?}{puntero(Nodo)/n, string/c, nat/i}{bool}{n $\neq$ NULL $\land$ i $\leq$ longitud(c)}
	\tadAxioma{definido?(nodo, clave, i)}
	{\IF i $<$ Longitud(clave) THEN
		$\neg$(nodo.hijos[ord(clave[i])] = NULL) $\yluego$ \\ definido?(nodo.hijos[ord(clave[i])], clave, i+1)
	ELSE
		nodo.contieneValor
	FI}


	~

	\tadOperacion{obtener}{puntero(Nodo)/n, string/c, nat/i}{$\alpha$}{n $\neq$ NULL $\land$ i $\leq$ longitud(c) $\yluego$ definido?(n, c, i)}

	\tadAxioma{obtener(nodo, clave, i)}
	{\IF i $<$ Longitud(clave) THEN
		obtener(nodo.hijos[ord(clave[i])], clave, i+1)
	ELSE
		nodo.valor
	FI}


	\TituloDis{Representación del iterador}

	\begin{Estructura}{Iterador DiccionarioString($\alpha$)}[itDicStr]

		\begin{Tupla}[itDicStr]
			\tupItem{itClaves}{itConj(String)}
			\tupItem{dic}{puntero(dicStr)}
		\end{Tupla}

	\end{Estructura}


	\TituloDis{Invariante de Representación del iterador}


	\begin{enumerate}

		\item El diccionario no es nulo

		\item El iterador de las claves corresponde a las claves del diccionario

	\end{enumerate}

	\Rep[itDicStr][it]{(it.dic $\neq$ NULL) $\yluego$ esPermutacion(SecuSuby(it.itClaves), (*it.dic).claves)}


	\TituloDis{Función de Abstracción del iterador}

	\Abs[itDicStr]{itBi(tupla(String, $\alpha$))}[it]{b}{Anteriores(b) $=$ Tuplas(Anteriores(it.itClaves), *it.dic) $\land$ \\ Siguientes(b) $=$ Tuplas(Siguientes(it.itClaves), *it.dic)}

	~

	\tadOperacion{Tuplas}{secu(String)/cs, dicc(String{,}\ $\alpha$)/dic}{secu(tupla(String, $\alpha$))}{}

	\tadAxioma{Tuplas(cs, dic)}
	{\IF vacia?(cs) THEN
		\secuvacia
	ELSE
		$\langle$ prim(cs), obtener(prim(cs), dic) $\rangle$ \puntito Tuplas(fin(cs), dic)
	FI}


\end{Representacion}

\begin{Algoritmos}

	\TituloDis{Algorítmos de DiccionarioString(\texorpdfstring{$\alpha$}{α})}

	\begin{algorithm}[H]
		\NoCaptionOfAlgo
		\caption{\algoritmo{iCrearDiccionario}{}{dicStr}}
		res.raiz $\leftarrow$ CrearNodo()\OdeLinea{1}
		res.claves $\leftarrow$ Vacio()\OdeLinea{1}
	\end{algorithm}

	\complejidad: $O(1)$

	\justifcomp: solo se crea un nodo vacío y el conjunto vacío de claves

	~

	\begin{algorithm}[H]
		\NoCaptionOfAlgo
		\caption{\algoritmo{CrearNodo}{}{puntero(Nodo)}}
		Nodo: nuevo\OdeLinea{1}
		nuevo.contieneValor $\leftarrow$ false\OdeLinea{1}
		\For(\OdeBloque{1}){$i \leftarrow 0$ \KwTo $255$}{ 
			nuevo.hijos[i] $\leftarrow$ NULL\OdeLinea{1}
		}
		res $\leftarrow$ \&nuevo\OdeLinea{1}
	\end{algorithm}

	\complejidad: $O(1)$

	\justifcomp: se asigna NULL a una cantidad fija de posiciones

	~

	\begin{algorithm}[H]
		\NoCaptionOfAlgo
		\caption{\algoritmo{iDefinir}{\Inout{dic}{dicStr}, \In{c}{String}, \In{s}{$\alpha$}}{}}
		puntero(Nodo): entrada $\leftarrow$ dic.raiz\OdeLinea{1}
		\For(\OdeBloque{|c_{max}|}){$i \leftarrow 0$ \KwTo Longitud(c)}{
			\If(\OdeBloque{1}){(*entrada).hijos[ord(c[i])] == NULL} {
				(*entrada).hijos[ord(c[i])] $\leftarrow$ CrearNodo()\OdeLinea{1}
			}
			entrada $\leftarrow$ (*entrada).hijos[ord(c[i])]\OdeLinea{1}
		}

		\If(\OdeBloque{1}){not (*entrada).contieneValor} {
			AgregarRapido(dic.claves, c)\OdeLinea{1}
			(*entrada).contieneValor $\leftarrow$ true\OdeLinea{1}
		}

		(*entrada).valor $\leftarrow$ s\OdeLinea{copy(s)}
	\end{algorithm}

	\complejidad: $O(|c_{max}| + copy(s))$

	\justifcomp: el ciclo itera hasta el largo de la clave a insertar y luego la copia para insertarla; la clave solo se agrega al conjunto si no estaba definida antes

	~

	\begin{algorithm}[H]
		\NoCaptionOfAlgo
		\caption{\algoritmo{iDefinido?}{\Inout{dic}{dicStr}, \In{c}{String}}{bool}}
		puntero(Nodo): entrada $\leftarrow$ dic.raiz\OdeLinea{1}
		\For(\OdeBloque{|c_{max}|}){$i \leftarrow 0$ \KwTo Longitud(c)}{
			\lIf(\OdeBloque{1}){entrada == NULL} {break}
			entrada $\leftarrow$ (*entrada).hijos[ord(c[i])]\OdeLinea{1}
		}

		% TODO podemos usar logica de cortocircuito? usamos simbolos de C para logica?
		res $\leftarrow$ not entrada == NULL \&\& (*entrada).contieneValor\OdeLinea{1}
	\end{algorithm}

	\complejidad: $O(|c_{max}|)$

	\justifcomp: el ciclo como máximo itera hasta el largo de la clave buscada

	~

	\begin{algorithm}[H]
		\NoCaptionOfAlgo
		\caption{\algoritmo{iObtener}{\Inout{dic}{dicStr}, \In{c}{String}}{$\alpha$}}
		puntero(Nodo): entrada $\leftarrow$ dic.raiz\OdeLinea{1}
		\For(\OdeBloque{|c_{max}|}){$i \leftarrow 0$ \KwTo Longitud(c)}{
			entrada $\leftarrow$ (*entrada).hijos[ord(c[i])]\OdeLinea{1}
		}

		res $\leftarrow$ (*entrada).valor\OdeLinea{copy(s)}
	\end{algorithm}

	\complejidad: $O(|c_{max}| + copy(s))$

	\justifcomp: el ciclo itera hasta el largo de la clave a obtener y luego la copia para retornarla

	~

	\begin{algorithm}[H]
		\NoCaptionOfAlgo
		\caption{\algoritmo{iBorrar}{\Inout{dic}{dicStr}, \In{c}{String}}{}}
		puntero(Nodo): entrada $\leftarrow$ dic.raiz\OdeLinea{1}
		\For(\OdeBloque{|c_{max}|}){$i \leftarrow 0$ \KwTo Longitud(c)}{
			entrada $\leftarrow$ (*entrada).hijos[ord(c[i])]\OdeLinea{1}
		}

		% TODO hace romper los nodos que no sirven?
		(*entrada).contieneValor $\leftarrow$ false\OdeLinea{1}
		Eliminar(dic.claves, c)\OdeLinea{|c_{max}| * \#c}
	\end{algorithm}

	\complejidad: $O(|c_{max}| * \#c)$

	\justifcomp: el ciclo itera hasta el largo de la clave a borrar y luego asigna un booleano, y borra la clave del conjunto

	\TituloDis{Algorítmos del iterador}

	\begin{algorithm}[H]
		\NoCaptionOfAlgo
		\caption{\algoritmo{iCrearIt}{\In{dic}{dicStr}}{itDicStr}}
		res.itClaves $\leftarrow$ CrearIt(dic.claves)\OdeLinea{1}
		res.dic $\leftarrow$ \&dic\OdeLinea{1}
	\end{algorithm}

	\complejidad: $O(1)$

	\justifcomp: solo se guarda un puntero al diccionario y se crea el iterador de las claves

	~

	\begin{algorithm}[H]
		\NoCaptionOfAlgo
		\caption{\algoritmo{iHaySiguiente}{\In{it}{itDicStr}}{bool}}
		res $\leftarrow$ HaySiguiente(it.itClaves)\OdeLinea{1}
	\end{algorithm}

	\complejidad: $O(1)$

	\justifcomp: comparte la complejidad del iterador de las claves

	~

	\begin{algorithm}[H]
		\NoCaptionOfAlgo
		\caption{\algoritmo{iHayAnterior}{\In{it}{itDicStr}}{bool}}
		res $\leftarrow$ HayAnterior(it.itClaves)\OdeLinea{1}
	\end{algorithm}

	\complejidad: $O(1)$

	\justifcomp: comparte la complejidad del iterador de las claves

	~

	\begin{algorithm}[H]
		\NoCaptionOfAlgo
		\caption{\algoritmo{iSiguiente}{\In{it}{itDicStr}}{tupla(String, $\alpha$)}}
		String: clave $\leftarrow$ Siguiente(it.itClaves)\OdeLinea{1}
		res $\leftarrow \langle$ clave, Obtener(*it.dic, clave) $\rangle$\OdeLinea{|c_{max}|}
	\end{algorithm}

	\complejidad: $O(|c_{max}|)$

	\justifcomp: comparte la complejidad del iterador de las claves, y luego consulta el diccionario a través de su función de obtener

	~

	\begin{algorithm}[H]
		\NoCaptionOfAlgo
		\caption{\algoritmo{iAnterior}{\In{it}{itDicStr}}{tupla(String, $\alpha$)}}
		String: clave $\leftarrow$ Anterior(it.itClaves)\OdeLinea{1}
		res $\leftarrow \langle$ clave, Obtener(*it.dic, clave) $\rangle$\OdeLinea{|c_{max}|}
	\end{algorithm}

	\complejidad: $O(|c_{max}|)$

	\justifcomp: comparte la complejidad del iterador de las claves, y luego consulta el diccionario a través de su función de obtener

	~

	\begin{algorithm}[H]
		\NoCaptionOfAlgo
		\caption{\algoritmo{iAvanzar}{\Inout{it}{itDicStr}}{}}
		it.itClaves $\leftarrow$ Avanzar(it.itClaves)\OdeLinea{1}
	\end{algorithm}

	\complejidad: $O(1)$

	\justifcomp: comparte la complejidad del iterador de las claves

	~

	\begin{algorithm}[H]
		\NoCaptionOfAlgo
		\caption{\algoritmo{iRetroceder}{\Inout{it}{itDicStr}}{}}
		it.itClaves $\leftarrow$ Retroceder(it.itClaves)\OdeLinea{1}
	\end{algorithm}

	\complejidad: $O(1)$

	\justifcomp: comparte la complejidad del iterador de las claves

\end{Algoritmos}