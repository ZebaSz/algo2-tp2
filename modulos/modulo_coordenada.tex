\section{Módulo Coordenada}

Descripcion TODO ¿CUALES TIENEN DESCRIPCION Y CUALES NO?

\subsection{Interfaz}

\textbf{se explica con}: \tadNombre{coordenada}.

\textbf{géneros}: \TipoVariable{coor}.

~

\subsubsection{Operaciones básicas de Coordenada}

\InterfazFuncion{CrearCoor}{\In {y,x}{nat}}{coor}
[true]
{$res$ $\igobs$ crearCoor(y,x)}
[$O(1)$]
[Creación de una coordenada con latitud $x$ y longitud $y$]
[]

~

\InterfazFuncion{Latitud}{\In{c}{coor}}{nat}
[true]
{$res \igobs$ latitud(c)}
[$O(1)$]
[Devuelve la latitud de la coordenada $c$]
[]

~

\InterfazFuncion{Longitud}{\In{c}{coor}}{nat}
[true]
{$res \igobs$ longitud(c)}
[$O(1)$]
[Devuelve la longitud de la coordenada $c$]
[]

~

\InterfazFuncion{DistEuclidea}{\In{c,d}{coor}}{nat}
[true]
{$res \igobs$ distEuclidea($c, d$)]}
[$O(1)$]
[Devuelve la distancia euclideana entre las coordenadas c y d.]
[]
~

\InterfazFuncion{CoordenadaArriba}{\In{c}{coor}}{coor}
[true]
{$res \igobs$ coordenadaArriba(c)}
[$O(1)$]
[Devuelve una coordenada con latitud de c + 1, y longitud de c como párametros]
[]

~

\InterfazFuncion{CoordenadaAbajo}{\In{c}{coor}}{coor}
[latitud(c) $>$ 0]
{$res \igobs$ coordenadaAbajo(c)}
[$O(1)$]
[Devuelve una coordenada con latitud de c - 1, y longitud de c como párametros]
[]

~

\InterfazFuncion{CoordenadaALaDerecha}{\In{c}{coor}}{coor}
[true]
{$res \igobs$ coordenadaALaDerecha(c)}
[$O(1)$]
[Devuelve una coordenada con latitud de c, y longitud de c + 1 como párametros]
[]

~

\InterfazFuncion{CoordenadaALaIzquierda}{\In{c}{coor}}{coor}
[longitud(c) $>$ 0]
{$res \igobs$ coordenadaALaIzquierda(c)}
[$O(1)$]
[Devuelve una coordenada con latitud de c, y longitud de c - 1 como párametros]
[]

~
%%TODO CLAVES

\subsubsection{Representación de Coordenada}

\begin{Estructura}{ Coordenada }[estr]
	\begin{Tupla}[estr]
		\tupItem{lat}{nat}
		\tupItem{long}{nat}
	\end{Tupla}

\end{Estructura}


\subsubsection{Invariante de Representación}

\Rep[estr][e]{
	true
	}

\subsubsection{Función de Abstracción}

\Abs[estr]{coor}[e]{c}{latitud(c) == e.lat $\land$ longitud(c) == e.long}

\subsection{Algoritmos}

\begin{algorithm}[H]
	\SetAlgoLined
	\NoCaptionOfAlgo
	\caption{\algoritmo{iCrearCoor}{\In{y}{nat}, \In{x}{nat}}{estr}}
	res.latitud $\leftarrow$ y\;
	res.longitud $\leftarrow$ x\;
\end{algorithm}

\begin{algorithm}[H]
	\SetAlgoLined
	\NoCaptionOfAlgo
	\caption{\algoritmo{iLatitud}{\In{e}{estr}}{nat}}
	res $\leftarrow$ e.latitud\;
\end{algorithm}

\begin{algorithm}[H]
	\SetAlgoLined
	\NoCaptionOfAlgo
	\caption{\algoritmo{iLongitud}{\In{e}{estr}}{nat}}
	res $\leftarrow$ e.longitud\;
\end{algorithm}


%%TODO preguntar si va con iCrearCoor o con CrearCoor

\begin{algorithm}[H]
	\SetAlgoLined
	\NoCaptionOfAlgo
	\caption{\algoritmo{iCoordenadaArriba}{\In{e}{estr}}{estr}}
	res $\leftarrow$ iCrearCoor(iLatitud(e)+1, iLongitud(e))\; 
\end{algorithm}

\begin{algorithm}[H]
	\SetAlgoLined
	\NoCaptionOfAlgo
	\caption{\algoritmo{iCoordenadaAbajo}{\In{e}{estr}}{estr}}
	res $\leftarrow$ iCrearCoor(iLatitud(e)-1, iLongitud(e))\; 
\end{algorithm}

\begin{algorithm}[H]
	\SetAlgoLined
	\NoCaptionOfAlgo
	\caption{\algoritmo{iCoordenadaALaDerecha}{\In{e}{estr}}{estr}}
	res $\leftarrow$ iCrearCoor(iLatitud(e), iLongitud(e)+1)\; 
\end{algorithm}

\begin{algorithm}[H]
	\SetAlgoLined
	\NoCaptionOfAlgo
	\caption{\algoritmo{iCoordenadaALaIzquierda}{\In{e}{estr}}{estr}}
	res $\leftarrow$ iCrearCoor(iLatitud(e), iLongitud(e)-1)\; 
\end{algorithm}

\begin{algorithm}[H]
	\SetAlgoLined
	\NoCaptionOfAlgo
	\caption{\algoritmo{iDistEuclidea}{\In{e1}{estr}, \In{e2}{estr}}{nat}}
	nat: la $\leftarrow$ 0\;
	nat: lo $\leftarrow$ 0\;
	\eIf{iLatitud(e1) $>$ iLatitud(e2)}{
		la $\leftarrow$ (iLatitud(e1) - iLatitud(e2)) $\times$ (iLatitud(e1) - iLatitud(e2))
	}{
		la $\leftarrow$ (iLatitud(e2) - iLatitud(e1)) $\times$ (iLatitud(e2) - iLatitud(e1))
	}
	\eIf{iLongitud(e1) $>$ iLongitud(e2)}{
		lo $\leftarrow$ (iLongitud(e1) - iLongitud(e2)) $\times$ (iLongitud(e1) - iLongitud(e2))
	}{
		lo $\leftarrow$ (iLongitud(e2) - iLongitud(e1)) $\times$ (iLongitud(e2) - iLongitud(e1))
	}
	res $\leftarrow$ la + lo\; 
\end{algorithm}