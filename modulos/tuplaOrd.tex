\section{Módulo Tupla con Orden(\texorpdfstring{$\alpha$}{α}, \texorpdfstring{$\beta$}{β})}

Este módulo permite crear tuplas con orden absoluto. El orden se define por el segundo elemento de la tupla, y en caso de coincidir se utiliza el primer elemento para desempatar.

\begin{Interfaz}

	\textbf{parámetros formales}\hangindent=2\parindent\\
	\parbox{1.7cm}{\textbf{géneros}}$\alpha$\\
	\parbox[t]{1.7cm}{\textbf{función}}\parbox[t]{.5\textwidth-\parindent-1.7cm}{
		\InterfazFuncion{$\bullet < \bullet$}{\In{a_1}{$\alpha$}, \In{a_2}{$\alpha$}}{bool}
		{$res \igobs (a_1 < a_2)$}
		[$\Theta(isLess(a_1, a_2))$]
		[comparación de $\alpha$'s]
	}
	\parbox[t]{1.7cm}{\textbf{función}}\parbox[t]{.5\textwidth-\parindent-1.7cm}{
		\InterfazFuncion{$\bullet < \bullet$}{\In{b_1}{$\beta$}, \In{b_2}{$\beta$}}{bool}
		{$res \igobs (b_1 < b_2)$}
		[$\Theta(isLess(b_1, b_2))$]
		[comparación de $\beta$'s]
	}\\[2ex]
	\parbox[t]{1.7cm}{\textbf{función}}\parbox[t]{.5\textwidth-\parindent-1.7cm}{
		\InterfazFuncion{$\bullet = \bullet$}{\In{a_1}{$\beta$}, \In{a_2}{$\beta$}}{bool}
		{$res \igobs (b_1 = b_2)$}
		[$\Theta(equal(b_1, b_2))$]
		[función de igualdad de $\beta$'s]
	}
	\parbox[t]{1.7cm}{\textbf{función}}\parbox[t]{.5\textwidth-\parindent-1.7cm}{
		\InterfazFuncion{Copiar}{\In{a}{$\alpha$}}{$\alpha$}
		{$res \igobs a$}
		[$\Theta(copy(a))$]
		[función de copia de $\alpha$'s]
	}\\[2ex]
	\parbox[t]{1.7cm}{\textbf{función}}\parbox[t]{.5\textwidth-\parindent-1.7cm}{
		\InterfazFuncion{Copiar}{\In{b}{$\beta$}}{$\beta$}
		{$res \igobs b$}
		[$\Theta(copy(b))$]
		[función de copia de $\beta$'s]
	}

	\textbf{se explica con}: \tadNombre{tupla($\alpha$, $\beta$)}.

	\textbf{géneros}: \TipoVariable{tupOrd($\alpha$, $\beta$)}.

	\TituloDis{Operaciones básicas de Tupla con Orden}

	\InterfazFuncion{CrearTupla}{\In{a}{$\alpha$}, \In{s}{$\beta$}}{tupOrd}
	{$res \igobs \langle a, s \rangle$}
	[$O(1)$]
	[crea una tupla.]
	[]	

	\InterfazFuncion{$\pi_1$}{\In{t}{tupOrd}}{$\alpha$}
	{$res \igobs \pi_1 (t)$}
	[$O(1)$]
	[Devuelve el primer componente de la tupla]

	\InterfazFuncion{$\pi_2$}{\In{t}{tupOrd}}{$\beta$}
	{$res \igobs \pi_2 (t)$}
	[$O(1)$]
	[Devuelve la segunda componente de la tupla]

	\InterfazFuncion{\puntito $<$ \puntito}{\In{t1}{tupOrd}, \In{t2}{tupOrd}}{$\alpha$}
	{$res \igobs t1 < t2$}
	[$O(isLess(t1.primer, t2.primer) + isLess(t1.segun, t2.segun) + equals(t1.segun, t2.segun))$]
	[Devuelve si la primer tupla es menor a la segunda]

	\TituloDis{Especificación de las operaciones auxiliares utilizadas en la interfaz}

	\begin{tad}{Tupla Extendida}
	\parskip=0pt
	\tadExtiende{\tadNombre{tupla($\alpha$, $\beta$)}}

	\textbf{parámetros formales}\hangindent=2\parindent\\
	\parbox{1.7cm}{\textbf{géneros}}$\alpha,\beta$\\
	\parbox{\parindent+1.7cm}{\textbf{operaciones}}\parbox[t]{\textwidth-\parindent-1.7cm}{
		\tadOperacionInline
		{\puntito $<$ \puntito}
		{$\alpha$,$\alpha$}
		{bool}\\
		\tadOperacionInline
		{\puntito $<$ \puntito}
		{$\beta$,$\beta$}
		{bool}\\
		\tadOperacionInline
		{\puntito $=$ \puntito}
		{$\beta$,$\beta$}
		{bool}
	}

	\tadTitulo{otras operaciones (exportadas)}

	\tadOperacion{\argumento $<$ \argumento}{tupOrd, tupOrd}{bool}{}
	
	\tadAxiomas
	\tadAxioma{a $<$ b}{$\pi_2$(a) $<$ $\pi_2$(b) $\lor$ ($\pi_1$(a) = $\pi_1$(b) $\land$ $\pi_1$(a) $<$ $\pi_1$(b))}


	\end{tad}

\end{Interfaz}


\begin{Representacion}
	\TituloDis{Representación de Tupĺa con Orden}

	\begin{Estructura}{tupla($\alpha$, $\beta$)}[eT]
		\begin{Tupla}[eT]
			\tupItem{primer}{$\alpha$}
			\tupItem{segun}{$\beta$}
		\end{Tupla}
	\end{Estructura}

	\TituloDis{Invariante de Representación}

	\RepFc[eT][e]{true}


	\TituloDis{Función de Abstracción}

	\Abs[eT]{tupOrd}[e]{t}{e.primer = $\pi_1 (t)$ $\land$ e.segun = $\pi_2 (t)$}
\end{Representacion}

\begin{Algoritmos}
	\TituloDis{Algorítmos de Tupla con Orden}

	\begin{algorithm}[H]
		\NoCaptionOfAlgo
		\caption{\algoritmo{iCrearTupla}{\In{a}{$\alpha$}, \In{s}{$\beta$}}{eT}}
		res.primer $\leftarrow$ a\OdeLinea{1}
		res.segun $\leftarrow$ s\OdeLinea{1}
	\end{algorithm}

	\complejidad: $O(1)$


	~

	\begin{algorithm}[H]
		\NoCaptionOfAlgo
		\caption{\algoritmo{i$\pi_1$}{\In{t}{eT}}{$\alpha$}}
		res $\leftarrow$ t.primer\OdeLinea{1}
	\end{algorithm}

	\complejidad: $O(1)$

	~

	\begin{algorithm}[H]
		\NoCaptionOfAlgo
		\caption{\algoritmo{i$\pi_2$}{\In{t}{eT}}{$\alpha$}}
		res $\leftarrow$ t.segun\OdeLinea{1}
	\end{algorithm}

	\complejidad: $O(1)$

	~

	\begin{algorithm}[H]
		\NoCaptionOfAlgo
		\caption{\algoritmo{\argumento $<$ \argumento}{\In{t1}{eT}, \In{t2}{eT}}{bool}}
		res $\leftarrow$ t1.segun $<$ t2.segun $\lor$ (t1.segun = t2.segun $\land$ t1.primer $<$ t2.primer)\OdeLinea{1}
	\end{algorithm}

	\complejidad: $O(isLess(t1.primer, t2.primer) + isLess(t1.segun, t2.segun) + equals(t1.segun, t2.segun))$

	\justifcomp: las comparaciones entre tuplas se realizan a través de sus componentes

\end{Algoritmos}