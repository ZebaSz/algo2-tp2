\section{Módulo ConjuntoOrd(\texorpdfstring{$\alpha$}{α})}

Este conjunto genérico se implementa sobre un arbol de búsqueda autobalanceado (AVL). Esto permite agregar, buscar y remover cualquier elemento en tiempo logarítmico. Se requiere que $\alpha$ tenga definida una relación de orden estricta.

Usaremos $\#c$ para denotar la cantidad de entradas del diccionario. También usaremos la función $isLess$ para denotar costos de comparación entre elementos.

\begin{Interfaz}

	\textbf{parámetros formales}\hangindent=2\parindent\\
	\parbox{1.7cm}{\textbf{géneros}}$\alpha$\\
	\parbox[t]{1.7cm}{\textbf{función}}\parbox[t]{.5\textwidth-\parindent-1.7cm}{
		\InterfazFuncion{$\bullet < \bullet$}{\In{a_1}{$\alpha$}, \In{a_2}{$\alpha$}}{bool}
		{$res \igobs (a_1 < a_2)$}
		[$\Theta(isLess(a_1, a_2))$]
		[comparación de $\alpha$'s]
		% TODO: ver todas las complejidades con esto
	}
	\parbox[t]{1.7cm}{\textbf{función}}\parbox[t]{.5\textwidth-\parindent-1.7cm}{
		\InterfazFuncion{Copiar}{\In{a}{$\alpha$}}{$\alpha$}
		{$res \igobs a$}
		[$\Theta(copy(a))$]
		[función de copia de $\alpha$'s]
	}

	\textbf{se explica con}: \tadNombre{Conjunto($\alpha$)}, \tadNombre{Iterador Unidireccional($\alpha$)}.

	\textbf{géneros}: \TipoVariable{conjOrd}, \TipoVariable{itConjOrd}.

	\TituloDis{Operaciones básicas de ConjuntoOrd($\alpha$)}

	\InterfazFuncion{Vacio}{}{conjOrd}
	{res $\igobs \emptyset$ }
	[$O(1)$]
	[crea el conjunto vacío.]

	\InterfazFuncion{Agregar}{\Inout{c}{conjOrd}, \In{a}{$\alpha$}}{}
	[$c \igobs c_0$]
	{$c \igobs$ Ag($a, c_0$)}
	[$O(log(\#c) \times isLess(\alpha_1, \alpha_2))$]
	[agrega $a$ al conjunto.]
	[se almacenan copias de $a$.]

	\InterfazFuncion{Pertenece?}{\In{c}{conjOrd}, \In{a}{$\alpha$}}{bool}
	{$res \igobs$ $a \in c$)]}
	[$O(log(\#c) \times isLess(\alpha_1, \alpha_2)) $]
	[devuelve \texttt{true} si $a$ esta en el conjunto.]

	\InterfazFuncion{Vacio?}{\In{c}{conjOrd}}{bool}
	{$res \igobs$ vacio?($c$)]}
	[$O(1)$]
	[devuelve \texttt{true} si el conjunto está vacio.]

	\InterfazFuncion{Borrar}{\Inout{c}{conjOrd}, \In{a}{$\alpha$}}{}
	[$c \igobs c_0$)]
	{$c \igobs$ borrar($a, c_0$)}
	[$O(log(\#c) \times isLess(\alpha_1, \alpha_2)) $]
	[elimina un elemento del conjunto si el mismo existía.]

	\InterfazFuncion{Minimo}{\Inout{c}{conjOrd}}{$\alpha$}
	[$\neg$vacio?($c$)]
	{$res \igobs$ minimo($c$)}
	[$O(log(\#c) \times isLess(\alpha_1, \alpha_2)) $]
	[devuelve el elemento mas chico del conjunto.]

	\TituloDis{Especificación de las operaciones auxiliares utilizadas en la interfaz}

	\begin{tad}{Conjunto($\alpha$) Extendido}
	\parskip=0pt
	\tadExtiende{\tadNombre{Conjunto($\alpha$)}}

	\tadTitulo{otras operaciones (exportadas)}
	\tadOperacion{minimo}{conj($\alpha$)/c}{$\alpha$}{$\neg\emptyset$?(c)}

	\tadAxiomas
	\tadAxioma{minimo(c)}{\IF $\#(c) = 1$ THEN 
		dameUno(c) 
	ELSE 
		{\IF dameUno(c) $<$ minimo(sinUno(c)) THEN
			dameUno(c)
		ELSE
			minimo(sinUno(c))
		FI}
	FI}

	~

	\end{tad}


	\TituloDis{Operaciones básicas del iterador}

	El iterador que presentamos no permite modificar el conjunto recorrido.

	\InterfazFuncion{CrearIt}{\In{c}{conjOrd}}{itConjOrd}
	{true}
	[$O(1)$]
	[crea un iterador unidireccional de los elementos.]
	[el iterador se invalida si y solo si se elimina el siguiente elemento del iterador. Además, siguientes($res$) podría cambiar completamente ante cualquier operación que modifique el conjunto.]

	\InterfazFuncion{HayMas}{\In{it}{itConjOrd}}{bool}
	{$res$ $\igobs$ hayMas?($it$)}
	[$O(1)$]
	[devuelve \texttt{true} si y sólo si en el iterador todavía quedan elementos para avanzar.]

	\InterfazFuncion{Actual}{\In{it}{itConjOrd}}{$\alpha$}
	[HayMas?($it$)]
	{$res$ $\igobs$ Actual($it$)}
	[$O(1))$]
	[devuelve el elemento siguiente a la posición del iterador.]

	\InterfazFuncion{Avanzar}{\Inout{it}{itConjOrd}}{}
	[$it = it_0$ $\land$ HayMas?($it$)]
	{$it$ $\igobs$ Avanzar($it_0$)}
	[$O(log(\#c))$]
	[avanza a la posición siguiente del iterador.]

\end{Interfaz}


\begin{Representacion}
	\TituloDis{Representación de conjOrd}

	\begin{Estructura}{conjOrd}[eConj]
		\begin{Tupla}[eConj]
			\tupItem{raiz}{puntero(Nodo)}
		\end{Tupla}

		~ 

		\begin{Tupla}[Nodo]
			\tupItem{hijoIzq}{puntero(Nodo)}
			\tupItem{hijoDer}{puntero(Nodo)}				
			\tupItem{elem}{$\alpha$}
			\tupItem{alto}{nat}
		\end{Tupla}
	\end{Estructura}

	\TituloDis{Invariante de Representación de ConjuntoOrd($\alpha$)}

	\begin{enumerate}
		
		\item No hay valores repetidos

		\item Para cualquier nodo del arbol, el valor de su hijo izquierdo es menor y el de su hijo derecho es mayor al suyo

		\item La altura de un Nodo es la altura del hijo con mayor altura mas 1

		\item El factor de balanceo es $\leq$ 1 (donde el factor de balanceo es el modulo de la diferencia de las alturas de los hijos)

	\end{enumerate}
	
	\Rep[eConj][e]{
		sinRepetidos(e.raiz, $\emptyset$)  $\land$  ABB?(e.raiz) $\land$ altoValido?(e.raiz)
	}	


	\TituloDis{Función de Abstracción}

	\Abs[eConj]{conjOrd}[e]{d}
	{($\forall a$: $\alpha$) a $\in$ d = Pertenece?(a, e)}

	\TituloDis{Operaciones auxiliares}

	\tadOperacion{sinRepetidos}{puntero(nodo), conj(nat)}{bool}{}
	\tadOperacion{ABB?}{puntero(Nodo)/nodo}{bool}{}
	\tadOperacion{menor?}{puntero(Nodo)/padre, puntero(Nodo)/hijo}{bool}{padre $\neq$ NULL}
	\tadOperacion{altoValido?}{puntero(nodo)}{bool}{}
	\tadOperacion{mayorAltura}{nat, nat}{nat}{}
	\tadOperacion{factorDeBalanceo}{nat, nat}{nat}{}
	\tadOperacion{cantidadHijos}{puntero(Nodo)}{nat}{}
	\tadOperacion{definido?}{$\alpha$, puntero(nodo)}{bool}{}
	~

	\tadAxioma{sinRepetidos(padre, elems)}{
		padre $=$ NULL $\oluego$(padre.elem $\not\in$ elems $\land$ \\
		sinRepetidos(padre.hijoIzq, Ag(padre.elem, c)) $\land$ \\
		sinRepetidos(padre.hijoDer, Ag(padre.elem, c)))
	}

	~

	\tadAxioma{ABB?(nodo)}{
		(nodo = NULL) $\oluego$ \\ 
		(menor?(nodo, (*nodo).hijoDer) $\land$ $\neg$menor?(nodo, (*nodo).hijoIzq) $\land$ \\
		ABB?((*nodo).hijoIzq) $\land$ ABB?((*nodo).hijoDer))
	}

	~

	\tadAxioma{menor?(padre, hijo)}{
		(hijo = NULL) $\oluego$ 
		padre.elem $<$ hijo.elem)
	}

	~

	\tadAxioma{altoValido?(nodo)}{
		nodo = NULL $\oluego$ \\
		((*nodo).alto = mayorAltura((*nodo).hijoIzq, (*nodo).hijoDer) + 1 $\land$ \\
		factorDeBalanceo(cantidadHijos((*nodo).hijoIzq), cantidadHijos((*nodo).hijoDer)) $\leq$ 1 
	}

	~

	\tadAxioma{mayorAltura(izq, der)}{
		{\IF cantidadHijos(izq) $<$ cantidadHijos(der) THEN
			cantidadHijos(izq)
		ELSE
			cantidadHijos(der)
		FI}
	}

	~

	\tadAxioma{cantidadHijos(nodo)}{
		{\IF nodo = NULL THEN
			0
		ELSE
			$\beta$((*nodo).hijoIzq $\neq$ NULL) + cantidadHijos((*nodo).hijoIzq) + \\
			$\beta$((*nodo).hijoDer $\neq$ NULL) + cantidadHijos((*nodo).hijoDer)		
		FI}
	}

	~

	\tadAxioma{factorDeBalanceo(izq, der)}{
		{\IF izq $<$ der THEN
			der - izq 
		ELSE
			izq - der
		FI}
	}

	~	

	\tadAxioma{definido?(a , n)}{
		n $\neq$ NULL $\yluego$ (n.elem = a $\lor$ definido?(j, n.hijoIzq) $\lor$ definido?(j, n.hijoDer))
	}

	\TituloDis{Representación del iterador}

	\begin{Estructura}{itConjOrd($\alpha$)}[eItConj]
		\begin{Tupla}[eItConj]
			\tupItem{conjunto}{puntero(conjOrd)}
			\tupItem{pila}{pila(puntero(Nodo))}
		\end{Tupla}
	\end{Estructura}

	\TituloDis{Invariante de Representación del iterador}

	\begin{enumerate}

		\item El conjunto no es nulo

		\item Ningún elemento de la pila es nulo

		\item Todo elemento de la pila es menor al anterior

	\end{enumerate}

	\Rep[eItConj][it]{it.conjunto $\neq$ NULL $\land$ PilaValida?(it.pila)}


	\TituloDis{Función de Abstracción del iterador}

	\Abs[eItConj]{itUni($\alpha$)}[it]{b}{Siguientes(b) $=$ Sigs(it.pila)}

	\tadOperacion{PilaValida?}{pila(puntero(Nodo))}{bool}{}
	\tadOperacion{Sigs}{pila(puntero(Nodo))}{secu($\alpha$)}{}
	\tadOperacion{Hijos}{puntero(Nodo)/n}{secu($\alpha$)}{n $\neq$ NULL}
	\tadOperacion{HijosDer}{pila(puntero(Nodo))/p}{secu($\alpha$)}{}

	~
	\tadAxioma{PilaValida?(p)}{
	vacia?(p) $\oluego$ (PilaValida?(desapilar(p)) $\land$ tope(p) $\neq$ NULL $\yluego$ \\
	(vacia?(desapilar(p)) $\oluego$ (*tope(p)).id $<$ (*tope(desapilar(p))).id))}

	~

	\tadAxioma{Sigs(p)}{
	\IF vacia?(p) THEN
		\secuvacia
	ELSE
		Hijos(tope(p)) \& HijosDer(desapilar(p))
	FI}


	\tadAxioma{Hijos(n)}{
	\IF n = NULL THEN
		\secuvacia
	ELSE
		(*n).id \puntito (Hijos((*n).hijoIzq) \& Hijos((*n).hijoDer))
	FI}

	~

	\tadAxioma{HijosDer(n)}{
	\IF vacia?(p) THEN
		\secuvacia
	ELSE
		Hijos((*tope(p)).hijoDer) \& HijosDer(desapilar(p))
	FI}


\end{Representacion}

\begin{Algoritmos}
	\TituloDis{Algoritmos de ConjuntoOrd($\alpha$)}

	\begin{algorithm}[H]
		\NoCaptionOfAlgo
		\caption{\algoritmo{iVacio}{}{eConj}}
		res.raiz $\leftarrow$ NULL\OdeLinea{1}
	\end{algorithm}

	\complejidad: $O(1)$

	\justifcomp: solo se inicializan los punteros como nulos.

	~

	\begin{algorithm}[H]
		\NoCaptionOfAlgo
		\caption{\algoritmo{iPertenece?}{\In{c}{eConj}, \In{a}{$\alpha$}}{bool}}
		res $\leftarrow$ Buscar(c.raiz, a) $\neq$ NULL\OdeLinea{log(\#c) \times isLess(\alpha_1, \alpha_2)}
	\end{algorithm}

	\complejidad: $ O(log(\#c)$ $\times$ isLess($ \alpha_1, \alpha_2 $))

	\justifcomp: la complejidad de buscar un elemento es de orden de la altura del arbol, que por invariante de AVL sabemos que es $log(\#c)$

	~

	\begin{algorithm}[H]
		\NoCaptionOfAlgo
		\caption{\algoritmo{iVacio?}{\In{c}{eConj}}{bool}}
		res $\leftarrow$ c.raiz $=$ NULL\OdeLinea{1}
	\end{algorithm}

	\complejidad: $O(1)$

	\justifcomp: solo se ingresa a la raiz del arbol

	~

	\begin{algorithm}[H]
		\NoCaptionOfAlgo
		\caption{\algoritmo{iAgregar}{\Inout{c}{eConj}, \In{a}{$\alpha$}}{}}
		\If{not Pertenece?(c, a)}{
			c.raiz $\leftarrow$ Insertar(a, c.raiz)\OdeLinea{log(\#c) \times isLess(\alpha_1, \alpha_2)}
		}
	\end{algorithm}

	\complejidad: $O(log(\#c) \times isLess(\alpha_1, \alpha_2))$

	\justifcomp: la complejidad de buscar un elemento es de orden de la altura del arbol, que por invariante de AVL sabemos que es $log(\#c)$

	~

	\begin{algorithm}[H]
		\NoCaptionOfAlgo
		\caption{\algoritmo{iBorrar}{\Inout{c}{eConj}, \In{a}{$\alpha$}}{}}
		c.raiz $\leftarrow$ Remover(a, c.raiz)\OdeLinea{log(\#c) \times isLess(\alpha_1, \alpha_2)}
	\end{algorithm}

	\complejidad: $O(log(\#c) \times isLess(\alpha_1, \alpha_2))$

	\justifcomp: la complejidad de buscar, agregar o eliminar un elemento es de orden de la altura del arbol, que por invariante de AVL sabemos que es $log(\#c)$

	~	

	\begin{algorithm}[H]
		\NoCaptionOfAlgo
		\caption{\algoritmo{iMinimo}{\Inout{c}{eConj}}{$\alpha$}}
		puntero(Nodo): minimo $\leftarrow$ c.raiz\OdeLinea{1}
		\While(\OdeBloque{log(\#c) \times isLess(\alpha_1, \alpha_2)}){(*minimo).hijoIzq $\neq$ NULL}{
				minimo $\leftarrow$ (*minimo.)hijoIzq \OdeLinea{1}
			}
		res $\leftarrow$ minimo.elem
	\end{algorithm}

	\complejidad: $O(log(\#c) \times isLess(\alpha_1, \alpha_2))$

	\justifcomp: hay que recorrer una rama del arbol hasta llegar al extremo izquierdo, y por invariante de AVL sabemos que es $log(\#c)$

	~

	\TituloDis{Algoritmos auxiliares}

	\begin{algorithm}[H]
		\NoCaptionOfAlgo
		\caption{\algoritmo{Buscar}{\In{a}{$\alpha$}, \In{nodo}{puntero(Nodo)}}{puntero(Nodo)}}
		\Pre{\texttt{true}}
		\Post{res es igual a un nodo hijo o al que viene por parametro, donde a es la clave del Nodo o bien la clave no está, en cuyo caso es NULL}
		\BlankLine
		\uIf{nodo = NULL \textbf{or} (*nodo).elem = a}{
			res $\leftarrow$ nodo\OdeLinea{1}
		}
		\uElseIf{a $<$ (*nodo).elem}{
			res $\leftarrow$ Buscar(a, (*nodo).hijoIzq)\OdeLinea{h}
		}
		\Else{
			res $\leftarrow$ Buscar(a, (*nodo).hijoDer)\OdeLinea{h}
		}
	\end{algorithm}

	~

	\begin{algorithm}[H]
		\NoCaptionOfAlgo
		\caption{\algoritmo{Insertar}{\In{a}{$\alpha$}, \Inout{nodo}{puntero(Nodo)}}{puntero(Nodo)}}
		\Pre{\texttt{true}}
		\Post{res es igual a la nueva raíz del subarbol que se pasó por parámetro con el nuevo elemento insertado y balanceado}
		\BlankLine
		\eIf{nodo = NULL}{
			res $\leftarrow$ CrearNodo(a, s)\OdeLinea{1}
		}{	
			\eIf{a $<$ n.elem}{
				(*nodo).hijoIzq $\leftarrow$ Insertar(a, s, (*nodo).hijoIzq)\OdeLinea{h}
			}{
				(*nodo).hijoDer $\leftarrow$ Insertar(a, s, (*nodo).hijoDer)\OdeLinea{h}
			}
			res $\leftarrow$ Balancear(nodo)\OdeLinea{1}
		}
	\end{algorithm}

	~

	\begin{algorithm}[H]
		\NoCaptionOfAlgo
		\caption{\algoritmo{Remover}{\In{a}{$\alpha$}, \In{nodo}{puntero(Nodo)}}{puntero(Nodo)}}
		\Pre{entre los nodo que desprenden del nodo del parametro se encuentra un nodo que contiene a $a$} 
		\Post{res es igual a la nueva raíz del subarbol que se pasó por parámetro pero ahora con el hijo removido}
		\BlankLine
		\uIf{nodo = NULL}{
			res $\leftarrow$ nodo\OdeLinea{1}
		}
		\uElseIf{a $<$ (*nodo).elem}{
			(*nodo).hijoIzq $\leftarrow$ Remover(a, (*nodo).hijoIzq)\OdeLinea{h}
			res $\leftarrow$ Balancear(nodo)\OdeLinea{1}
		}
		\uElseIf{a $>$ (*nodo).elem}{
			(*nodo).hijoDer $\leftarrow$ Remover(a, (*nodo).hijoDer)\OdeLinea{h}
			res $\leftarrow$ Balancear(nodo)\OdeLinea{1}
		}
		\Else{
			puntero(Nodo): i $\leftarrow$ (*nodo).hijoIzq\OdeLinea{1}
			puntero(Nodo): d $\leftarrow$ (*nodo).hijoDer\OdeLinea{1}
			\eIf{d = NULL}{
				res $\leftarrow$ i\OdeLinea{1}
			}{
				puntero(Nodo): minimo $\leftarrow$ BuscarMinimo(d)\OdeLinea{h}
				minimo.hijoDer $\leftarrow$ RemoverMinimo(d)\OdeLinea{h}
				minimo.hijoIzq $\leftarrow$ i\OdeLinea{1}
				res $\leftarrow$ Balancear(minimo)\OdeLinea{1}
			}
		}
	\end{algorithm}


	~

	\begin{algorithm}[H]
		\NoCaptionOfAlgo
		\caption{\algoritmo{CrearNodo}{\In{a}{$\alpha$}}{puntero(Nodo)}}
		\Pre{\texttt{true}}
		\Post{$res$ apunta a un nodo de alto 1 con valor $a$}
		\BlankLine
		Nodo: nuevo\OdeLinea{1}
		nuevo.hijoIzq $\leftarrow$ NULL\OdeLinea{1}
		nuevo.hijoDer $\leftarrow$ NULL\OdeLinea{1}
		nuevo.alto $\leftarrow$ 1\OdeLinea{1}
		nuevo.elem $\leftarrow$ a\OdeLinea{copy(\alpha)}
		res $\leftarrow$ \&nuevo\OdeLinea{1}
	\end{algorithm}
	
	\complejidad: $O(copy(\alpha))$

	\justifcomp: solo se crea e inicializa un nodo con una copia del valor.

	~

	\begin{algorithm}[H]
		\NoCaptionOfAlgo
		\caption{\algoritmo{ArreglarAlto}{\Inout{nodo}{puntero(Nodo)}}{}}
		\Pre{la altura de los hijos de $nodo$ es correcta}
		\Post{la altura de $nodo$ es igual a la altura máxima de sus hijos + 1}
		\BlankLine
		nat: alturaIzq $\leftarrow$ Altura((*nodo).hijoIzq)\OdeLinea{1}
		nat: alturaDer $\leftarrow$ Altura((*nodo).hijoDer)\OdeLinea{1}
		\eIf{alturaIzq $<$ alturaDer}{
			(*nodo).alto $\leftarrow$ alturaDer + 1\OdeLinea{1}
		}{
			(*nodo).alto $\leftarrow$ alturaIzq + 1\OdeLinea{1}
		}
	\end{algorithm}
	
	\complejidad: $O(1)$

	\justifcomp: solo se realizan comparaciones para las alturas de los hijos

	~

	\begin{algorithm}[H]
		\NoCaptionOfAlgo
		\caption{\algoritmo{Altura}{\In{nodo}{puntero(Nodo)}}{nat}}
		\Pre{\texttt{true}}
		\Post{$res$ es 0 si $nodo$ es nulo, o igual a la altura del nodo al que apunta en caso contrario}
		\BlankLine
		\eIf{nodo = NULL}{
			res $\leftarrow$ 0\OdeLinea{1}
		}{
			res $\leftarrow$ (*nodo).alto\OdeLinea{1}
		}
	\end{algorithm}
	
	\complejidad: $O(1)$

	\justifcomp: solo se verifica si es nulo o se lee un componente.

	~

	\begin{algorithm}[H]
		\NoCaptionOfAlgo
		\caption{\algoritmo{Balancear}{\Inout{nodo}{puntero(Nodo)}}{puntero(Nodo)}}
		\Pre{$nodo \neq$ NULL}
		\Post{el módulo del factor de balanceo de $res$ es menor a 2}
		\BlankLine
		ArreglarAlto(nodo)\OdeLinea{1}
		nat: alturaIzq $\leftarrow$ Altura((*nodo).hijoIzq)\OdeLinea{1}
		nat: alturaDer $\leftarrow$ Altura((*nodo).hijoDer)\OdeLinea{1}
		\uIf{alturaDer $>$ alturaIzq \textbf{and} alturaDer - alturaIzq = 2}{
			\If{Altura((*nodo).hijoDer.hijoIzq) $>$ Altura((*nodo).hijoDer.hijoDer)}{
				(*nodo).hijoDer $\leftarrow$ rotarDer((*nodo).hijoDer\OdeLinea{1}
			}
			res $\leftarrow$ rotarIzq(nodo)\OdeLinea{1}
		}
		\uElseIf{alturaIzq $>$ alturaDer \textbf{and} alturaIzq - alturaDer = 2}{
			\If{Altura((*nodo).hijoIzq.hijoDer) $>$ Altura((*nodo).hijoIzq.hijoIzq)}{
				(*nodo).hijoIzq $\leftarrow$ rotarIzq((*nodo).hijoIzq)\OdeLinea{1}
			}
			res $\leftarrow$ rotarDer(nodo)\OdeLinea{1}
		}
		\Else{
			res $\leftarrow$ nodo\OdeLinea{1}
		}
	\end{algorithm}

	~

	\begin{algorithm}[H]
		\NoCaptionOfAlgo
		\caption{\algoritmo{rotarDer}{\In{nodo}{puntero(Nodo)}}{puntero(Nodo)}}
		\Pre{$nodo$ tiene un hijo izquierdo}
		\Post{$res$ es el hijo izquierdo de nodo, y arbol se rota a la derecha}
		\BlankLine
		puntero(nodo) : aux $\leftarrow$ (*nodo).hijoIzq\OdeLinea{1}
		(*nodo).hijoIzq $\leftarrow$ (*aux).hijoDer\OdeLinea{1}
		(*aux).hijoDer $\leftarrow$ nodo\OdeLinea{1}
		ArreglarAlto(nodo)\OdeLinea{1}
		ArreglarAlto(aux)\OdeLinea{1}
		res $\leftarrow$ aux\OdeLinea{1}
	\end{algorithm}

	~

	\begin{algorithm}[H]
		\NoCaptionOfAlgo
		\caption{\algoritmo{rotarIzq}{\In{nodo}{puntero(Nodo)}}{puntero(Nodo)}}
		\Pre{$nodo$ tiene un hijo derecho}
		\Post{$res$ es el hijo derecho de nodo, y arbol se rota a la izquierda}
		\BlankLine
		puntero(nodo) : aux $\leftarrow$ (*nodo).hijoDer\OdeLinea{1}
		(*nodo).hijoDer $\leftarrow$ aux.hijoIzq\OdeLinea{1}
		aux.hijoIzq $\leftarrow$ nodo\OdeLinea{1}
		ArreglarAlto(nodo)\OdeLinea{1}
		ArreglarAlto(aux)\OdeLinea{1}
		res $\leftarrow$ aux\OdeLinea{1}
	\end{algorithm}
	
	\complejidad: $O(1)$

	\justifcomp: las rotaciones son una serie de comparaciones y asignaciones de punteros ($\Theta(1)$). El invariante de orden del arbol de búsqueda (ABB) se preserva luego de cada rotación. El invariante de balanceo de AVL se restaura para cada subarbol al final de \texttt{Balancear}, pero no luego de cada rotación individual (puede ser necesario rotar un subarbol de manera temporalmente imbalanceada para restaurar el balance de un arbol).

	\TituloDis{Algoritmos del iterador}

	\begin{algorithm}[H]
		\NoCaptionOfAlgo
		\caption{\algoritmo{CrearIt}{\In{c}{eConj}}{eItConj}}
		res.conjunto $\leftarrow$ \&(c)\OdeLinea{1}
		res.pila $\leftarrow$ Vacía()\OdeLinea{1}
		\If{\textbf{not} c.raiz = NULL}{
			Apilar(res.pila, c.raiz)\OdeLinea{1}
		}
	\end{algorithm}

	\complejidad: $O(1)$

	\justifcomp: solo se inicializa el iterador.

	~

	\begin{algorithm}[H]
		\NoCaptionOfAlgo
		\caption{\algoritmo{HayMas}{\In{it}{eItConj}}{bool}}
		res $\leftarrow$ EsVacía?(it.pila)\OdeLinea{1}
	\end{algorithm}

	\complejidad: $O(1)$

	\justifcomp: solo se utilizan operaciones básicas de pila.

	~

	\begin{algorithm}[H]
		\NoCaptionOfAlgo
		\caption{\algoritmo{Actual}{\In{it}{eItConj}}{jugador}}
		res $\leftarrow$ (*Tope(it.pila)).id\OdeLinea{1}
	\end{algorithm}

	\complejidad: $O(1)$

	\justifcomp: solo se utilizan operaciones básicas de pila.

	~

	\begin{algorithm}[H]
		\NoCaptionOfAlgo
		\caption{\algoritmo{Avanzar}{\Inout{it}{eItConj}}{}}
		\While(\OdeBloque{1}){\textbf{not} EsVacía(it.pila)} {
			puntero(Nodo) : tope $\leftarrow$ Tope(it.pila)\OdeLinea{1}
			\eIf{not (*tope).hijoIzq = NULL}{
				Apilar(it.pila, (*tope).hijoIzq)\OdeLinea{1}
					\textbf{break}\OdeLinea{1}
			}{
				Desapilar(it.pila)\OdeLinea{1}
				\If{not (*tope).hijoDer = NULL}{
					Apilar(it.pila, (*tope).hijoDer)\OdeLinea{1}
					\textbf{break}\OdeLinea{1}
				}
			}
		}
	\end{algorithm}

	\complejidad: $O(1)$

	\justifcomp: la pila contiene aquellas posiciones para las cuales todavía no se visitó el nodo derecho, y en el caso del tope, tampoco se revisó el nodo izquierdo, además de estar ordenadas por posición en el arbol (las primeras que se agregan correpsonden a posiciones superiores).

	El peor caso posible para un ABB normal es que la pila contenga todos los elementos del conjunto (ningún nodo del arbol tiene nodo derecho).

	En el caso particular de los AVL la complejidad es constante porque, por el invariante de factores de balance, la cantidad de posiciones que tenemos que recorrer "hacia arriba" para encontrar un nodo derecho es como máximo 2. Por ende, la cantidad de ciclos a realizar tiene máximo constante y no depende del tamaño del arbol.

\end{Algoritmos}