\section{Módulo Chupense una Pija}
	Description %%TODO habria que decir que la raiz es el elemento minimo

\begin{Interfaz}
	\textbf{se explica con}: \tadNombre{CONJ?}. %%TODO

	\textbf{géneros}: \TipoVariable{chupaPij}.

	\TituloDis{Operaciones básicas de Chupense una Pija}

	\InterfazFuncion{Vacia}{}{chupaPij}
	[]
	{$res \igobs \emptyset$}
	[]
	[]
	[]

	\InterfazFuncion{Agregar}{\In{a}{RELLENAR}, \Inout{h}{chupaPij}}{}
	[]
	{$h \igobs$ Ag($a, h_0$)}
	[]
	[]
	[]

	\InterfazFuncion{Vacia?}{\In{h}{chupaPij}}{bool}
	{$res \igobs$ $\emptyset$?($h$)]}
	[]
	[]

	\InterfazFuncion{Pertenece?}{\In{a}{RELLENAR}, \In{h}{chupaPij}}{bool}
	{$res \igobs$ a $\in$ h]}
	[]
	[]	

	\InterfazFuncion{Borrar}{\In{a}{RELLENAR}, \Inout{h}{chupaPij}}{}
	[$h \igobs h_0$]
	{$h \igobs$ $h_0 - a$}
	[]
	[]
	[]

	%%TODO EXTENDER TAD
	\InterfazFuncion{Menor}{\In{a}{RELLENAR}, \Inout{h}{chupaPij}}{RELLENAR}
	[]
	{}
	[]
	[]
	[]

\end{Interfaz}


\begin{Representacion}
	\TituloDis{Representación de chupaPij}

	\begin{Estructura}{ pija }[ePij]
		\begin{Tupla}[ePij]
			\tupItem{raiz}{puntero(Nodo)}
		\end{Tupla}
		\begin{Tupla}[Nodo]
			\tupItem{hijoIzq}{puntero(Nodo)}
			\tupItem{hijoDer}{puntero(Nodo)}				
			\tupItem{cantPokemons}{nat}
			\tupItem{idJugador}{nat}
			\tupItem{alto}{nat}
		\end{Tupla}
	\end{Estructura}

	\TituloDis{Invariante de Representación}

	\begin{enumerate}
		
		\item No hay idJugador repetidos

		\item Los nodos de la izquierda tienen mayor prioridad

		\item Los nodos de la derecha tienen menor prioridad

		\item La altura de un Nodo es la altura del hijo con mayor altura mas 1

		\item Los factores son -1, 0 o +1

	\end{enumerate}
	

	\Rep[ePij][e]{
		sinRepetidos(e.raiz, $\emptyset$) $\land$ ABB?(e.raiz) $\land$ factorValido?(e.raiz)
	}


	\tadOperacion{sinRepetidos}{puntero(nodo), conj(nat)}{}
	\tadOperacion{ABB?}{puntero(Nodo)/nodo}{bool}{}
	\tadOperacion{menor?}{puntero(Nodo)/padre, puntero(Nodo)/hijo}{bool}{padre $\neq$ NULL}
	\tadOperacion{altoValido?}(puntero(nodo)){bool}{}
	\tadOperacion{mayorAltura}{nat, nat}{nat}{}
	\tadOperacion{factorDeBalanceo}{nat, nat}{nat}{}
	\tadOperacion{cantidadHijos}{puntero(Nodo)}{}

 	~

	\tadAxioma{sinRepetidos(padre, ids)}{
		padre $\neq$ NULL $\oluego$( padre.idJugador $\not\in$ ids $\land$ \\
		sinRepetidos(padre.hijoIzq, Ag(padre.idJugador, c)) $\land$ \\
		sinRepetidos(padre.hijoDer, Ag(padre.idJugador, c)))
	}

	~

	\tadAxioma{ABB?(nodo)}{
		(nodo = NULL) $\oluego$ \\ 
		(menor?(nodo, nodo.hijoIzq) $\land$ $\neq$menor?(nodo, nodo.hijoDer) $\land$ \\
		ABB?(nodo.hijoIzq) $\land$ ABB?(nodo.hijoDer))
	}

	~

	\tadAxioma{menor?(padre, hijo)}{
		(hijo = NULL) $\oluego$ padre.cantPokemons $<$ hijo.cantPokemons $\lor$ \\
		(padre.cantPokemons = hijo.cantPokemons $\land$ padre.idJugador $<$ hijo.idJugador)
	}

	~

	\tadAxioma{altoValido?(nodo)}{
		nodo = NULL $\oluego$ \\
		(nodo.alto = mayorAltura(nodo.hijoIzq, nodo.hijoDer) + 1 $\land$ \\
		factorDeBalanceo(cantidadHijos(nodo.hijoIzq), cantidadHijos(nodo.hijoDer)) $\leq$ 1 
	}

	~

	\tadAxioma{mayorAltura(izq, der)}{
		{\IF cantidadHijos(izq) $<$ cantidadHijos(der) THEN
			cantidadHijos(izq)
		ELSE
			cantidadHijos(der)
		FI}
	}

	~

	\tadAxioma{cantidadHijos(nodo)}{
		{\IF nodo = NULL THEN
			0
		ELSE
			$\beta$(nodo.hijoIzq $\neq$ NULL) + cantidadHijos(nodo.hijoIzq) + \\
			$\beta$(nodo.hijoDer $\neq$ NULL) + cantidadHijos(nodo.hijoDer)		
		FI}
	}

	~

	\tadAxioma{factorDeBalanceo(izq, der)}{
		{\IF izq $<$ der THEN
			der - izq 
		ELSE
			izq - der
		FI}
	}

	\TituloDis{Función de Abstracción}

	\Abs[ePij]{chupaPij}[e]{c}{
		($\forall$ a: RELLENAR) a $\in$ c $\iff$ a $\in$ conjunto(c)
	} 

	\tadOperacion{conjunto}{puntero(Nodo)}{bool}{}
	\tadAxioma{conjunto(nodo)}{
		{\IF nodo = NULL THEN
			$\emptyset$
		ELSE
			Ag(nodo.idJugador, (conjunto(nodo.hijoIzq) $\cup$ conjunto(nodo.hijoDer)))	
		FI}
	}

\end{Representacion}

\begin{Algoritmos}
	\TituloDis{Algorítmos de Chupense una Pija}

	\begin{algorithm}[H]
		\NoCaptionOfAlgo
		\caption{\algoritmo{iVacia}{}{chupaPij}}
		res.raiz $\leftarrow$ NULL\;
	\end{algorithm}


	% \begin{algorithm}[H]
	% 	\NoCaptionOfAlgo
	% 	\caption{\algoritmo{Buscar}{\In{a}{RELLENAR}, \Inout{hijo}{puntero(Nodo)}}}
	% 	\While{hijo $\neq$ NULL $\yluego$ hijo.idJugador $\neq$ a}{
	% 		\eIf{hijo.cant $<$ a.cant $\lor$ (hijo.cant = a.cant $\land$ hijo.id $<$ a.id}{
	% 			hijo $\leftarrow$ hijo.hijoIzq\;
	% 		}{
	% 			hijo $\leftarrow$ hijo.hijoDer\;
	% 		}
	% 	}
	% \end{algorithm}

	\begin{algorithm}[H]
		\NoCaptionOfAlgo
		\caption{\algoritmo{iAgregar}{\In{a}{RELLENAR}, \Inout{h}{chupaPij}}{}}
	\end{algorithm}

	\begin{algorithm}[H]
		\NoCaptionOfAlgo
		\caption{\algoritmo{CrearNodo}{\In{a}{RELLNAR}}{puntero(Nodo)}}
		Nodo: nuevo\;
		nuevo.valor $\leftarrow$ a\; %%TODO CAMBIAR
		nuevo.hijoIzq $\leftarrow$ NULL\;
		nuevo.hijoDer $\leftarrow$ NULL\;
		nuevo.alto $\leftarrow$ 1\;
		res $\leftarrow$ \&nuevo\;
	\end{algorithm}

	\begin{algorithm}[H]
		\NoCaptionOfAlgo
		\caption{\algoritmo{Insertar}{\In{a}{RELLENAR}, \Inout{nodo}{puntero(Nodo)}}{puntero(Nodo)}}
		
		\eIf{nodo = NULL}{
			res $\leftarrow$ CrearNodo(a)\;
		}{	
			%%TODO hacer bien la comapracion
			\eIf{n.valor $<$ a}{
				nodo.hijoIzq $\leftarrow$ Insertar(a, nodo.hijoIzq)\;
			}{
				nodo.hijoDer $\leftarrow$ Insertar(a, nodo.hijoDer)\;
			}
			res $\leftarrow$ Balancear(nodo)\;
		}
	\end{algorithm}

	\begin{algorithm}[H]
		\NoCaptionOfAlgo
		\caption{\algoritmo{Balancear}{\Inout{nodo}{puntero(Nodo)}}{puntero(Nodo)}}
		%%TODO ArreglarFactor(nodo)
		\eIf(FactorDeBalanceo(nodo) = 2){
			\If(FactorDeBalanceo(nodo.hijoDer) $<$ 0){
				nodo.hijoDer $\leftarrow$ rotarDer(nodo.hijoDer);
			}
			res $\leftarrow$ rotarIzq(nodo)\;
		}{
			\eIf(FactorDeBalanceo(nodo) = -2){
				\If(FactorDeBalanceo(nodo.hijoDer) $>$ 0){
					nodo.hijoIzq $\leftarrow$ rotarIzq(nodo.hijoIzq);
				}
				res $\leftarrow$ rotarDer(nodo)\;
			}{
				res $\leftarrow$ nodo\;
			}
		}
	\end{algorithm}


	\begin{algorithm}[H]
		\NoCaptionOfAlgo
		\caption{\algoritmo{iVacia?}{\In{h}{chupaPij}}{bool}}
	\end{algorithm}

	\begin{algorithm}[H]
		\NoCaptionOfAlgo
		\caption{\algoritmo{iPertenece?}{\In{a}{RELLENAR}, \In{h}{chupaPij}}{bool}}
	\end{algorithm}

	\begin{algorithm}[H]
		\NoCaptionOfAlgo
		\caption{\algoritmo{iBorrar}{\In{a}{RELLENAR}, \Inout{h}{chupaPij}}{}}
	\end{algorithm}

	\begin{algorithm}[H]
		\NoCaptionOfAlgo
		\caption{\algoritmo{iMenor}{\In{a}{RELLENAR}, \Inout{h}{chupaPij}}{RELLENAR}}
	\end{algorithm}


\end{Algoritmos}